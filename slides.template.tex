\documentclass{beamer}
\usepackage[utf8]{inputenc}
\usepackage[T1]{fontenc}

\usefonttheme{professionalfonts}
\usepackage{fontspec}

\usetheme{MSU_Light}

%%Colors can be redfined here -- These are the names that should be overriden to re-style various parts of the theme
\definecolor{MSU_BKGCOLOR}{HTML}{97A2A2}
\definecolor{MSU_TITLECOLOR}{HTML}{18453B}
\definecolor{MSU_TEXTCOLOR}{HTML}{000000}
\definecolor{MSU_HIGHLIGHTCOLOR}{HTML}{CB5A28}


\setsansfont[
	Path = /Users/luke/Library/Fonts/,
	UprightFont = *-Light,
 	BoldFont = *-SemiBold,
 	ItalicFont = *-LightItalic]{Montserrat}

\newfontfamily\normalfont{Montserrat}[Path = /Users/luke/Library/Fonts/,
	UprightFont = *-Light,
 	BoldFont = *-SemiBold,
 	ItalicFont = *-LightItalic]
\newfontfamily\titlefont{Montserrat}[Path = /Users/luke/Library/Fonts/,
	UprightFont = *-Bold]
\newfontfamily\headingfont{Montserrat}[Path = /Users/luke/Library/Fonts/,
	UprightFont = *-SemiBold]

\normalfont\selectfont

\title{A title}
\date{\today}
\subtitle{Meeting name, Location}
\author[S. name]{\textbf{Author~Authorson}, An.~Authorson}

\begin{document}

\begin{frame}
\titlepage
\end{frame}

\begin{frame}
\tableofcontents
\end{frame}

\section{Examples}
\subsection{Text area}
\begin{frame}{Text area}
\begin{tikzoverlayarea}
  \begin{tikztextarea}{0.5\textwidth}
    \begin{itemize}
      \item List.
      \item Sub list:
      \begin{itemize}
        \item And another thing
      \end{itemize}
      \item {\tiny Can make things small too.}
      \item \raisebox{0.2em}{\tiny But might want to raise it a smidge.}
      \item Some more text.
    \end{itemize}
  \end{tikztextarea}
\end{tikzoverlayarea}
\end{frame}

\subsection{Positioning grid}
\begin{frame}{Positioning grid}
\begin{tikzoverlayarea}
  \drawareagrid
\end{tikzoverlayarea}
\end{frame}

\subsection{Arrows}
\begin{frame}{Arrows}
\begin{tikzoverlayarea}
  \drawareagrid
  \inserttikzarrow{red,very thick, -latex}{1}{4}{3.25}{2.0}
  \inserttikznode{red}{1}{4.5}{(1,4)}
  \inserttikznode{red}{3.25}{2.5}{(3.25,2)}

  %For more complicated arrows and paths, you can write the TikZ directly
  \draw[blue,-latex, very thick] (4,4) to [out=90,in=180] (6,6);
\end{tikzoverlayarea}
\end{frame}

\subsection{Graphics}
\begin{frame}{Graphics}
\begin{tikzoverlayarea}
  \drawareagrid
  \inserttikzpicturerotatefillwhite{1}{3}{2cm}{MSU_design_elements/helmet_green.eps}
  \inserttikzpicturefillwhite{5}{5}{4cm}{MSU_design_elements/helmet_green.eps}
  \inserttikzpicture{0}{6}{3cm}{MSU_design_elements/helmet_green.eps}
\end{tikzoverlayarea}
\end{frame}

\subsection{Text nodes}
\begin{frame}{Text nodes}
\begin{tikzoverlayarea}
  \drawareagrid
  \inserttikznode{font=\bf}{3}{3}{Some text}
  \inserttikznode{MSU_13,font=\bf}{6}{4}{Some other text}
  %For more complicated text nodes, you can write the TikZ directly

  \node[blue,font=\headingfont] at (10,5) [anchor=south west, rotate=-90] {$y=mx+c$};
\end{tikzoverlayarea}
\end{frame}

\definecolor{MSU_HIGHLIGHTCOLOR}{HTML}{000000}
\section{Slide re-styling}
\begin{frame}{Slide re-styling}
\begin{tikzoverlayarea}
  \begin{tikztextarea}{0.5\textwidth}
    Highlights and bottom branding should appear black for this slide!
  \end{tikztextarea}
  \inserttikzpicturefillwhite{8.5}{4}{4cm}{MSU_design_elements/helmet_green.eps}
\drawbottomoverlayblack
\end{tikzoverlayarea}
\end{frame}

\definecolor{MSU_HIGHLIGHTCOLOR}{HTML}{CB5A28}

\begin{frame}
\thanksshield
\end{frame}

\begin{frame}
\backups
\end{frame}

\end{document}
